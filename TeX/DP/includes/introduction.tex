World around us is ever changing and dynamic. To effectively synthesize and analyze such systems, we resort to mathematical modeling.
Mathematical models help us understand the behavior of systems, predict their responses to various inputs, and design control strategies to achieve desired outcomes.

Within this thesis we will be focusing on a single system, analyze it from the perspective of identification and control design.
Identifying the system's inputs, outputs, dynamics and static characteristics. To better illustrate the process hidden behind simple terminology of system identification and control design.
We will attempt to cover the entire process from the ground up, starting from system analysis, model derivation, identification, controller design, simulation, implementation and finally comparison of the different approaches used throughout the thesis.
The goal is to provide a comprehensive guide of the entire process of system identification, controller design, and validation, such that the reader can apply these concepts to other systems in the future.
The methods used within the thesis will mainly focus on non-linear system identification and non-linear control design, as these methods are more general and can be applied to a wider range of systems, without the limitation of linearization around a certain operating point (\acrshort{op}).
Experiments will be proposed and conducted to gather necessary data for the identification process. Designing correct experiments is crucial to ensure that the data collected represents the systems behavior accurately.
Furthermore, the process of deriving a physical model of the system can be done without having the collected data. However, having experimental data allows for validation of the derived model and ensures that it accurately represents the real-world system.

With this in mind, we will provide a simple yet effective methodology for deriving a non-linear model of a pendulum type system using Lagrangian mechanics, which can further be used in various similar systems; for example robotic arms, inverted pendulums, and other mechanical systems with rotational dynamics.
We will focus on deriving a non-linear state-space (\acrshort{nss}) model, show how to linearize it around a certain \acrshort{op}, giving us the linearized state-space (\acrshort{ss}) model. Discretization of both models will be covered.
Which comes in handy when designing discrete controllers to be implemented on microcontroller units (\acrshort{mcu}). We will design various discrete controllers, including but not limited to linear quadratic regulator (\acrshort{lqr}) and model predictive control (\acrshort{mpc}).
Simulations will be conducted to validate the derived models and designed controllers. Finally, the designed controllers will be implemented on an \acrshort{mcu}, and their performance will be compared to the simulation results, to demonstrate the difference between theoretical and practical applications.
